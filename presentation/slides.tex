% $Header: /cvsroot/latex-beamer/latex-beamer/solutions/conference-talks/conference-ornate-20min.en.tex,v 1.6 2004/10/07 20:53:08 tantau Exp $

\documentclass{beamer}

% This file is a solution template for:

% - Talk at a conference/colloquium.
% - Talk length is about 20min.
% - Style is ornate.

\mode<presentation>
{
  \usetheme{Boadilla}
  % or ...

  \setbeamercovered{transparent}
  % or whatever (possibly just delete it)
}

\title{Persistent Skiplists}

\subtitle{For Efficient Point Set Queries}

\author{Simon Pratt}
% - Give the names in the same order as the appear in the paper.
% - Use the \inst{?} command only if the authors have different
%   affiliation.

\institute[Carleton University]
{
  Carleton University
}
% - Use the \inst command only if there are several affiliations.
% - Keep it simple, no one is interested in your street address.

\date{August 30th, 2011}
% - Either use conference name or its abbreviation.
% - Not really informative to the audience, more for people (including
%   yourself) who are reading the slides online

\subject{Computer Science}
% This is only inserted into the PDF information catalog. Can be left
% out. 

% If you have a file called "university-logo-filename.xxx", where xxx
% is a graphic format that can be processed by latex or pdflatex,
% resp., then you can add a logo as follows:

% \pgfdeclareimage[height=0.5cm]{university-logo}{university-logo-filename}
% \logo{\pgfuseimage{university-logo}}

% If you wish to uncover everything in a step-wise fashion, uncomment
% the following command: 

%\beamerdefaultoverlayspecification{<+->}

\begin{document}

\begin{frame}
  \titlepage
\end{frame}

% Structuring a talk is a difficult task and the following structure
% may not be suitable. Here are some rules that apply for this
% solution: 

% - Exactly two or three sections (other than the summary).
% - At *most* three subsections per section.
% - Talk about 30s to 2min per frame. So there should be between about
%   15 and 30 frames, all told.

% - A conference audience is likely to know very little of what you
%   are going to talk about. So *simplify*!
% - In a 20min talk, getting the main ideas across is hard
%   enough. Leave out details, even if it means being less precise than
%   you think necessary.
% - If you omit details that are vital to the proof/implementation,
%   just say so once. Everybody will be happy with that.

\section{Motivation}

\subsection{Point Set Queries}

\begin{frame}
  \frametitle{Point Set Queries}
  % - A title should summarize the slide in an understandable fashion
  %   for anyone how does not follow everything on the slide itself.

  \begin{itemize}
  \item
    HighestNE
  \item
    LeftMostNE
  \item
    Highest3Sided
  \item
    Enumerate3Sided
  \end{itemize}
\end{frame}

\begin{frame}
  \frametitle{Applications}
  % - A title should summarize the slide in an understandable fashion
  %   for anyone how does not follow everything on the slide itself.

  \begin{itemize}
  \item
    TODO applications
  \end{itemize}
\end{frame}

\subsection{Previous Work}

\begin{frame}
  \frametitle{In-Place Priority Search Trees}
  \begin{itemize}
  \item
    Minati, Anil, Subhas, Michiel
  \item
    Built implementation
  \end{itemize}
\end{frame}

\begin{frame}
  \frametitle{Persistence}

  \begin{itemize}
  \item
    Planar Point Location using Persistent Search Trees\\
    Sarnak, Tarjan
  \end{itemize}
\end{frame}

\begin{frame}
  \frametitle{Skiplists}

  \begin{itemize}
  \item
    Skiplists
  \end{itemize}
\end{frame}

\section{My Results/Contribution}

\subsection{Algorithms}

\begin{frame}
  \frametitle{HighestNE}

  \begin{itemize}
  \item
    Outline 
  \item
    Time complexity
  \end{itemize}
\end{frame}

\begin{frame}
  \frametitle{LeftMostNE}

  \begin{itemize}
  \item
    Outline 
  \item
    Time complexity
  \end{itemize}
\end{frame}

\begin{frame}
  \frametitle{Highest3Sided}

  \begin{itemize}
  \item
    Outline 
  \item
    Time complexity
  \end{itemize}
\end{frame}

\begin{frame}
  \frametitle{Enumerate3Sided}

  \begin{itemize}
  \item
    Outline 
  \item
    Time complexity
  \end{itemize}
\end{frame}

\subsection{Persistent Skiplist Design}

\begin{frame}
  \frametitle{Persistent Skiplist Design}

  \begin{itemize}
  \item
    Path Copying
  \item
    Fat Nodes
  \item
    Limited Path Copying with Fixed-Size Fat Nodes
  \end{itemize}
\end{frame}

\subsection{Space Complexity}

\begin{frame}
  \frametitle{Space Complexity}

  \begin{itemize}
  \item
    Bounded by amortized space complexity
  \item
    Potential for operations
  \item
    Potential for sequences
  \end{itemize}  
\end{frame}

\section*{Summary}

\begin{frame}
  \frametitle<presentation>{Summary}

  % Keep the summary *very short*.
  \begin{itemize}
  \item
    The \alert{first main message} of your talk in one or two lines.
  \item
    The \alert{second main message} of your talk in one or two lines.
  \item
    Perhaps a \alert{third message}, but not more than that.
  \end{itemize}

\end{frame}

% All of the following is optional and typically not needed. 
\appendix
\section<presentation>*{\appendixname}
\subsection<presentation>*{For Further Reading}

\begin{frame}[allowframebreaks]
  \frametitle<presentation>{For Further Reading}

  \begin{thebibliography}{10}
    
  \beamertemplatebookbibitems
  % Start with overview books.

  \bibitem{Morin11}
    P. Morin
    \newblock {\em Open Data Structures}.
    \newblock Version 0.0 pre $\alpha$ : COMP2402 Fall 2011
    
  \beamertemplatearticlebibitems
  % Followed by interesting articles. Keep the list short. 

  \bibitem{DeMaheshwariNandySmid11}
    M. De, A. Maheshwari, S. C. Nandy, M. Smid.
    \newblock An In-Place Priority Search Tree
    \newblock
    {
      \em Proceedings of the 23rd Canadian Conference on Computational Geometry
    },
    331--336, August 2011.

  \bibitem{TarjanSarnak86}
    N. Sarnak and R. E. Tarjan.
    \newblock Planar Point Location Using Persistent Search Trees
    \newblock {\em Communications of the ACM},
    29(7):669--679, July 1986.

  \bibitem{Pugh90}
    W. Pugh.
    \newblock Skip lists: A probabilistic alternative to balanced trees.
    \newblock {\em Communications of the ACM},
    33(6):668--676, 1990.

  \end{thebibliography}
\end{frame}

\end{document}


