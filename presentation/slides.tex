% $Header: /cvsroot/latex-beamer/latex-beamer/solutions/conference-talks/conference-ornate-20min.en.tex,v 1.6 2004/10/07 20:53:08 tantau Exp $

\documentclass{beamer}

% This file is a solution template for:

% - Talk at a conference/colloquium.
% - Talk length is about 20min.
% - Style is ornate.

\mode<presentation>
{
  \usetheme{Boadilla}
  % or ...

  \setbeamercovered{transparent}
  % or whatever (possibly just delete it)
}

\title{Persistent Skip Lists}

\subtitle{For Efficient Point Set Queries}

\author{Simon Pratt}
% - Give the names in the same order as the appear in the paper.
% - Use the \inst{?} command only if the authors have different
%   affiliation.

\institute[Carleton University]
{
  Carleton University
}
% - Use the \inst command only if there are several affiliations.
% - Keep it simple, no one is interested in your street address.

\date{August 30th, 2011}
% - Either use conference name or its abbreviation.
% - Not really informative to the audience, more for people (including
%   yourself) who are reading the slides online

\subject{Computer Science}
% This is only inserted into the PDF information catalog. Can be left
% out. 

% If you have a file called "university-logo-filename.xxx", where xxx
% is a graphic format that can be processed by latex or pdflatex,
% resp., then you can add a logo as follows:

% \pgfdeclareimage[height=0.5cm]{university-logo}{university-logo-filename}
% \logo{\pgfuseimage{university-logo}}

% If you wish to uncover everything in a step-wise fashion, uncomment
% the following command: 

%\beamerdefaultoverlayspecification{<+->}

\begin{document}

\section{Title}
\begin{frame}
  \titlepage
\end{frame}

% Structuring a talk is a difficult task and the following structure
% may not be suitable. Here are some rules that apply for this
% solution: 

% - Exactly two or three sections (other than the summary).
% - At *most* three subsections per section.
% - Talk about 30s to 2min per frame. So there should be between about
%   15 and 30 frames, all told.

% - A conference audience is likely to know very little of what you
%   are going to talk about. So *simplify*!
% - In a 20min talk, getting the main ideas across is hard
%   enough. Leave out details, even if it means being less precise than
%   you think necessary.
% - If you omit details that are vital to the proof/implementation,
%   just say so once. Everybody will be happy with that.

\section{Motivation}

\subsection{Point Set Queries}

\begin{frame}
  \frametitle{Point Set Queries}

  \begin{itemize}
  \item
    Given a set of points in $\Re^2$
  \item
    Perform efficient 2 and 3 sided queries on the set of points
  \end{itemize}
\end{frame}

\begin{frame}
  \frametitle{Example Point Set Queries}
  % - A title should summarize the slide in an understandable fashion
  %   for anyone how does not follow everything on the slide itself.

  LeftMostNE

  \begin{itemize}
  \item
    Return the left most point bounded by $y_{min}$ and $x_{min}$
  \end{itemize}

    Enumerate3Sided

  \begin{itemize}
  \item
    Return all points bounded by $y_{min}$, $x_{min}$ and $x_{max}$
  \end{itemize}

\end{frame}

\subsection{Previous Work}

\begin{frame}
  \frametitle{In-Place Priority Search Trees}

  \begin{itemize}
  \item
    In-Place Data Structure which provides the same queries
  \item
    Binary tree
  \item
    Root node is highest y-coordinate
  \item
    Left and right subtrees are split along the median x-coordinate of
    the rest of the points
  \end{itemize}

\end{frame}

\begin{frame}
  \frametitle{Persistence}

  \begin{itemize}
  \item
    A new version of the data structure is saved upon each update
  \item
    Old versions of the data structure are queryable
  \item
    Some data structures work well for queries on a point set in $\Re$
  \item
    Can extend such a data structure to $\Re^2$ using persistence
  \end{itemize}

\end{frame}

\begin{frame}
  \frametitle{Skip List!}

  \begin{center}
    \includegraphics[width=300px]{figs/skiplist.pdf}
  \end{center}

\end{frame}

\section{Persistent Skip Lists}

\subsection{Solution}

\begin{frame}
  \frametitle{Persistent Skip List Design}

  \begin{itemize}
  \item
    Path Copying
  \item
    Fat Nodes
  \item
    Limited Path Copying with Fixed-Size Fat Nodes
  \end{itemize}

\end{frame}

\begin{frame}
  \frametitle{Performing Point Queries}

  \begin{itemize}
  \item
    An array A of points sorted by y-coordinate
  \item
    So the skiplist at a given timestamp corresponds to all of the points at or
    above the point at location timestamp in A
  \end{itemize}

\end{frame}

\begin{frame}
  \frametitle{Build Time}

  Given a randomly arranged array of unique points:
  \begin{itemize}
  \item
    O(logn) to sort points by y coordinate
  \item
    O(logn) to insert a single point into the skiplist
  \item
    O(nlogn) to perform n O(logn) insertions
  \end{itemize}

\end{frame}

\subsection{Algorithms}

\begin{frame}
  \frametitle{LeftMostNE}

  \begin{itemize}
  \item
    binary search on A to find lowest point higher than y bound
  \item
    skip list traversal to find left most point to the right of x bound
  \item
    Time complexity: O(logn)
  \end{itemize}

\end{frame}

\begin{frame}
  \frametitle{Enumerate3Sided}

  \begin{itemize}
  \item
    binary search on A to find lowest point higher than y bound
  \item
    skip list traversal to find left most point to the right of left x bound
  \item
    traverse lowest list linearly reporting all encountered points
    until finding the first point to the right of the right x bound
  \item
    Time complexity O(logn + k), where k is the number of points in the 
    query region
  \end{itemize}
\end{frame}

\subsection{Space Complexity}

\begin{frame}
  \frametitle{Space Complexity}

  \begin{itemize}
  \item
    Amortized space cost is the number of changes to the structure plus the
    change in potential
  \item
    Amortized space cost of an update is O(1)
  \item
    The sum of all updates have space cost of O(n)
  \item
    This is an upper bound to the space complexity
  \end{itemize}  
\end{frame}

\section*{Summary}

\begin{frame}
  \frametitle<presentation>{Summary: Persistent Skip Lists}

  \begin{center}
    % Keep the summary *very short*.
    \begin{itemize}
    \item
      \alert{logarithmic} query time
    \item
      \alert{linear} space
    \item
      \alert{O(nlogn)} build time
    \end{itemize}
  \end{center}

\end{frame}

\begin{frame}
  \begin{center}
    {\bf Questions?}
  \end{center}
\end{frame}

% All of the following is optional and typically not needed. 
\appendix
\section*{Appendix}

\subsection*{For Further Reading}

\begin{frame}
  \frametitle<presentation>{Bibliography}

  \begin{thebibliography}{10}
    
  \beamertemplatebookbibitems
  % Start with overview books.

  \bibitem{Morin11}
    P. Morin
    \newblock {\em Open Data Structures}.
    \newblock Version 0.0 pre $\alpha$ : COMP2402 Fall 2011
    
  \beamertemplatearticlebibitems
  % Followed by interesting articles. Keep the list short. 

  \bibitem{DeMaheshwariNandySmid11}
    M. De, A. Maheshwari, S. C. Nandy, M. Smid.
    \newblock An In-Place Priority Search Tree
    \newblock
    {
      \em Proceedings of the 23rd Canadian Conference on Computational Geometry
    },
    331--336, August 2011.

  \bibitem{TarjanSarnak86}
    N. Sarnak and R. E. Tarjan.
    \newblock Planar Point Location Using Persistent Search Trees
    \newblock {\em Communications of the ACM},
    29(7):669--679, July 1986.

  \bibitem{Pugh90}
    W. Pugh.
    \newblock Skip lists: A probabilistic alternative to balanced trees.
    \newblock {\em Communications of the ACM},
    33(6):668--676, 1990.

  \end{thebibliography}
\end{frame}

\end{document}


